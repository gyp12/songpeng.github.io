%%%%%%%%%%%%%%%%%%%%%%%%%%%%%%%%%%%%%%%%%
% Medium Length Professional CV
% LaTeX Template
% Version 2.0 (8/5/13)
%
% This template has been downloaded from:
% http://www.LaTeXTemplates.com
%
% Original author:
% Trey Hunner (http://www.treyhunner.com/)
%
% Important note:
% This template requires the resume.cls file to be in the same directory as the
% .tex file. The resume.cls file provides the resume style used for structuring the
% document.
%
%%%%%%%%%%%%%%%%%%%%%%%%%%%%%%%%%%%%%%%%%

%----------------------------------------------------------------------------------------
%	PACKAGES AND OTHER DOCUMENT CONFIGURATIONS
%----------------------------------------------------------------------------------------

\documentclass{resume} % Use the custom resume.cls style

\usepackage[left=0.75in,top=0.6in,right=0.75in,bottom=0.6in]{geometry} % Document margins
\usepackage{multicol}
\usepackage{enumitem}

\setlist{nolistsep}
\setlist[itemize]{leftmargin=*}
\name{Songpeng Zu} % Your name
\address{FIT 1-108, Tsinghua University, Beijing, China} % Your address
%\address{123 Pleasant Lane \\ City, State 12345} % Your secondary addess (optional)
\address{zsp07@mails.tsinghua.edu.cn \\ https://github.com/songpeng} % Your phone number and email

\begin{document}

%----------------------------------------------------------------------------------------
%	EDUCATION SECTION
%----------------------------------------------------------------------------------------

\begin{rSection}{Education}

{\bf Tsinghua University} \hfill {\em 2011.09 - Present} \\
PhD Candidate, Bioinformatics Laboratory, Department of Automation, Tsinghua University\\
Graduate Major GPA: 90.84/100
\vspace{-3mm}
\begin{multicols}{2}
%\vspace{-\topsep}
  \begin{itemize}
%\setlength{\parskip}{0pt}
%\setlength{\itemsep}{0pt plus 1pt}
\item Applied Stochastic Process
\item Real Analysis, Basic Functional Analysis
\item Pattern Recognition, Probabilistic Graph
\item Applied Abstract Algebra
\item Design of Bioinformatics Algorithms
\item Probability and Statistics in High Dimensions
  \end{itemize}
%\vspace{-\topsep}
\end{multicols}
\vspace{-3mm}

%\vspace{-\topsep}
{\bf Tsinghua University} \hfill {\em 2007.09 - 2011.07} \\
{Bachelor, School of Life Science, Tsinghua University}\hfill {Undergraduate Major GPA: 86.32/100}
%\begin{multicols}{2}
%  \begin{itemize}
%  \item Adavanced Calculus, Linear Algebra
%  \item Molecular Biology, Cell Biology
%  \item Biochemistry, Genetics, Biostatistics
%  \item General Physics (require for Physics major)
%  \item Organic Chemicstry, Physical Chemistry
%  \end{itemize}
%\end{multicols}
\end{rSection}

%----------------------------------------------------------------------------------------
%	WORK EXPERIENCE SECTION
%----------------------------------------------------------------------------------------
%----------------------------------------------------------------------------------------
\begin{rSection}{Basic Skills}
  {\bf Basic Machine Learning }\\
Have the basic understanding on matchine learning.
  \vspace{-3mm}
  \begin{itemize}
\item Kernel method, support vector machine, and Gauss process
\item Decision tree model, random forest and the Boosting method
\item Regression analysis, such as ridge regression, LASSO, and generalized linear model
\item Unsupervised learning,  such as K-means, principle component analysis
\item Probabilistic graph model, such as Bayesian network, Markov random field, and conditional random field
\item Neutral network model, restricted Boltzman machine, and deep learning model
  \end{itemize}
{\bf Basic Statistical Inference}\\
Have the basic training on statistical inference.
\vspace{-3mm}
\begin{itemize}
\item Monte Carlo strategies, such rejection sampling, Gibbs samplings, and Hamilton Monte Carlo
\item Bayesian statistics, such as Bayesian regression, and hierarchical Bayesian model
\item Statistical inference, such as maximum likelihood, hypothesis testing, and EM algorithm
\end{itemize}
{\bf Basic Convex Optimization}\\
Have the basic understanding on convex optimization.
\vspace{-3mm}
\begin{itemize}
\item Augmented lagrange method, proximal minimization, such as alternating direction method of multipliers
\item Newton or quasi Newton method, genetic algorithm, and simulation annealing
\end{itemize}
\end{rSection}

\begin{rSection}{Research Experience}
{\bf Department of Automation, Tsinghua University} \hfill {\em 2011.07 - Present}\\
Major on predicting compound-protein interactions from the machine learning perspective.
\vspace{-3mm}
\begin{itemize}
\item Predicting chemogenomic features from drug-target interactions by EM algorithm.
\item Predicting drug-target interactions by a graph partition model.
\item Quantitatively predicting on compound-protein interactions based on transfer learning.
\end{itemize}
{\bf Department of Statistics, Harvard University} \hfill {\em 2014.03 - 2014.09} \\
cis-eQTLs detection on GTEx Project.
\vspace{-3mm}
\begin{itemize}
\item Using the Bayesian Nonparametric tests via sliced inverse modeling to detect the non-linear relationships.
\end{itemize}
{\bf Internship in Baidu Inc, Beijing, China} \hfill {\em 2014.01 - 2014.02}
\vspace{-3mm}
\begin{itemize}
\item  Relating the influenza epidemics with the query data in China.
\item  Studying the AIDS risks in different regions with the query data in China.
\end{itemize}
{\bf Internship in Disease Control and Prevention Center, Liuzhou, China} \hfill {\em 2013.07 - 2013.08}
\vspace{-3mm}
\begin{itemize}
\item Studying the AIDS subtypes in Liuzhou by the DNA sequence data.
\item Helping them to construct the methods on sequence analysis of AIDS.
\end{itemize}
\end{rSection}

\begin{rSection}{Publication}
{\bf Zu S.}, Chen T, Li S. {\it Global optimization-based inference of chemogenomic features from drug-target interactions.} Bioinformatics, 2015.
(published online)
\end{rSection}

\begin{rSection}{Work Experience}
  \begin{itemize}
  \item {TA of Introduction to Systems Biology for undergraduate students} \hfill {\em 2014.09 - 2015.01}
%  \item {Internship in BAIDU Inc., Beijing, China} \hfill {\em 2014.01 - 2014.02}
  \item {TA of Probabilistic Graphical Models for graduate students} \hfill {\em 2013.09 - 2014.01}
  \item {Undergraduate Affair Counselor (for scholarship and financial aid assessment)} \hfill {\em 2011.08 - 2013.01}
%  \item {Complete the full Marathon} \hfill {\em 2009.10, 2010.10, 2011.10}
  \item {The volunteer of 2008 Beijing Olympic Games} \hfill {\em 2008.08}
  \end{itemize}
\end{rSection}
%----------------------------------------------------------------------------------------
%	TECHNICAL STRENGTHS SECTION
%----------------------------------------------------------------------------------------
\begin{rSection}{AWARDS}
  \begin{itemize}
  \item {Tsinghua Scholarship for Overseas Graduate Studies} \hfill {\em 2014}
  \item {Tsinghua Excellent Undergraduate Affair Counselor} \hfill {\em 2013}
  \item {Tsinghua Zhongying Tang Scholarship} \hfill {\em 2008, 2009, 2010}
  \end{itemize}
\end{rSection}
\begin{rSection}{Techniques and Interests}
\begin{tabular}{ @{} >{\bfseries}l @{\hspace{6ex}} l }
Computer Languages & R, Python, Perl, C/C++, Shell\\
Tools & Emacs, Vim, Latex \\
Interests & The international ballroom dance, Football
\end{tabular}

\end{rSection}

%----------------------------------------------------------------------------------------
%	EXAMPLE SECTION
%----------------------------------------------------------------------------------------

%\begin{rSection}{Section Name}

%Section content\ldots

%\end{rSection}

%----------------------------------------------------------------------------------------

\end{document}
